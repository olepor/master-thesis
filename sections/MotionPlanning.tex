\chapter{Motion Planning}
\label{sec:MotionPlanning}

% Alternative outlay: Overview -> Representation (discrete, cont, determ, prob,
% state-space implici/explicit)

\section{Overview/Preliminaries}
Motion planning is the task of manipulating a robot's configurations so that,
given an initial and goal state or posture, the planner (black box), is able to
create a sequence of actions that gives a feasible or optimal path through the
overarching planning environment, usually referred to as the world space. Thus a
motion planner can be seen as a machine which given an the input: a world, and
initial and goal states, produces a sequence of actions to move the robot from
its initial to the desired goal configuration. This plan is then passed on to
the trajectory generator in the system. Generally, planners can be separated
into complete and non-complete planners, meaning that given enough time, all
motion planning problems are solvable, only the solution is NP-Hard. Thus
feasible solutions will have to make compromise, and a lot of planners in use
today are what is called probabilistically complete, meaning that they converge
to a solution given infinite time. As such time-frames are of no use to us, we
generally have to settle for an approximation bound. There is a difference
between online and off-line motion planning, whereas the offline algorithm plans
in a static environment, the online algorithm is run continuously. However an
online algorithm can be simulated by running an offline algorithm repeatedly for
short intervals of time. However, this comes with the drawback, that no
guarantee can be made for completing a certain task.

\subsection{Building Blocks}

\subsection{The Robot and World Model}

The modelled world \(\mathcal{W}\), and eveything in it can be modelled
geometrically using any number of methods -- most usually polygons or some other
convex gemoetric structure. Thus the robot can also be modelled as a collection
of gemoetric structures \(\mathcal{A}\). Thus in order to move the robot in the
world space, rigid body transformations are applied. Given the robot as a subset
of the world space, the image of a robot transformation is then given as:
% \[
%   h(\mathcal{A}) = \set{h(a) \in \mathcal{W} \mid a \in \mathcal{A}} \\
%   h \colon \mathcal{A} \rightarrow \mathcal{W} \\
%   \mathcal{A} \and h(\mathcal{A}) \subset \mathcal{W} \\
% \]

The world as we model it contains two kinds of objects

\begin{itemize}
\item \textbf{Robot:} Body that is modeled geometrically and is controllable via
  a motion plan.
\item \textbf{Obstacles:} Portions of the world that are occupied by something
  other than the robot.
\end{itemize}

\subsection{Rigid Transformations}
\label{sec:rigidtransformations}


% \subsection{State Space}
% \label{subsec:State}
% Planning problems involve a state space. A state space is a set of all the
% possible configurations a model can be in at any given time. The state space
% can be either discrete or continuous, and can be represented either implicitly
% or explicitly.

\subsubsection{Configuration Space}
\label{sec:configuration-space}
The configuration space is a general abstraction used as a model for a wide
variety of motion planning problems. It is a manifold that arise from the
transformations applied to the robot. If the robot has n-degrees of freedom, the
set of transformations is usually a manifold of dimension n. This manifold is
called the configuration space of the robot, and is shortened \(\modelconfigurationspace{}\).
Thus in order to solve a motion planning problem, a search must be performed in
the configuration space. Thus the motion planning problem is now made into a
question of finding the best path to traverse the given manifold\cite{Lav06}.
(characterize and describe the configuration space structure.) In our case the
configuration space can be modeled as a topological space of the type \(SE(3)\),
meaning special-Euclidean three.

Using generalized coordinates, the configuration of a robot can be modeled as a
vector with n variables for the position in the configuration space, in
literature this space is usually denoted by \(\modelconfigurationspace{}\). As is most common,
the robot is modeled as a rigid body, in which two bodies cannot overlap, thus
points in the configuration space is separated into two sets. One set for which
the robot configuration overlaps with another object in the configuration space,
and is called \(\modelconfigurationspaceobst{}\), and given that the world space is
represented as \(\modelconfigurationspace{}\), the collision free space is the given as
,\(\modelconfigurationspacefree = \modelconfigurationspace \setminus \modelconfigurationspaceobst\). The obstacle
space can be represented in a number of ways, but we will stick to polygonal
models of the obstacle space.

\subsection{Action Space}
The action space is the set of actions that can be applied at any given state
the robot is located in. Thus one can model the action space as a function of
the robot's state. e.g.
\[
  U(x) = \set{u \in U \mid U(x) \neq \emptyset }
\]

\subsection{Initial and Goal States}
It is normal to define an initial state and a goal state as a starting and an
ending point of a planning problem. Where both the initial, and goal states can
be sets of states, meaning that it is not necessary to arrive specifically at
the target point. A starting point consisting of multiple configuration states
usually infers some kind of uncertainty in the robot model. The end goal can be
reached in one of two ways, wherein the two types characterize two different
forms of planning.
\begin{enumerate}
\item\textbf{Feasible:} Feasible planning has no concerns with optimality, and
  only concerns itself with finding a plan that arrives at the goal state.
\item\textbf{Optimal:} Optimal planning optimizes a feasible plan in some
  specified manner, with respect to some specified cost function.
\end{enumerate}

\subsection{A Plan}
If we are dealing with a fully deterministic model, a plan can simply be the set
of actions applied at each state in order to reach a goal state specified for
the problem at hand. However, if some measure of uncertainty is added to the
model, then literature describes this as planning in the \textit{information
  space}, and future states cannot be predicted exactly, thus some kind of
feedback model is used in order to plan in an uncertain environment. If we are
merely looking at solving the problem, a feasible plan is good enough, which
means a solution has to be found, but multiple solutions do not have to be found
and compared. However, if time, energy or some other parameters are to be
optimized, then a cost function has to be added into the planning model, in
order to evaluate which plan scores best when measured up against the others.

\subsection{Model}

\subsubsection{Discrete}
let \(\mathcal{X}\) be the discrete state space, and \(\ mathcal{U}(x)\) be the
set of actions available at each point \(x \in \mathcal{X}\). state transition:
\[
  x_{k+1} = f(x_k, u_k)
\]

\subsubsection{Continuous}

\subsubsection{Path and Trajectory}

The motion plan takes the form of a path, or a trajectory. This is represented
as a function \(\phi(\alpha) \colon [0,1] \rightarrow \mathcal{X}\), where
\(\mathcal{X}\) is the configuration space of the vehichle. If the
control-execution time is considered, then the explicit model of vehicle
kinematics and/or dynamics, as well as the dynamics of the possible obstacles.
Then the trajectory is represented as a time-parametrized function of the kind
\(\pi(t) \colon [0,T] \rightarrow \mathcal{X}\), where \(T\) is the planning
horizon. Unlike the path, the trajectory describes how the configuration of the
vehicle evolves over time.

Initial and Goal States. A path is a function f \ldots

\subsection{Time}
\label{subsec:Time}

Discrete vs Continuous

\subsection{Actions}
\label{subsec:Actions}

A plan is a sequence of actions that manipulate the state of the robot.

A planning problem involves a sequence of decisions made over time.


car-like model can look something like:

\section{Algorithms}
In general there are algorithms for which every motion planning problem can be
solved explicitly, only their running time is on the order of \(O(!n)\) , and
thus not feasible in practical applications. In the practical case it is
therefore normal to employ methods of approximation.

\subsection{Grid-based Search}
\hyphenation{grid-based-search}
\label{subsec:gridbasedsearch}

\subsection{Interval-based Search}
\hyphenation{interval-bases-search}
\label{subsec:intervalbasedsearch}

\subsection{Geometric Algorithms}
\hyphenation{geometric-algorithms}
\label{subsec:geometricalgorithms}

\subsection{Artificial Potential Fields}
\hyphenation{artificial-potential-fields}
\label{subsec:artificialpotentialfields}

\subsection{Sampling-based}
\label{subsec:samplingbased}









\section{Piano Mover's Problem}

\section{Discrete vs Continuous}

\section{Sampling}

\section{State Space}

\section{Complexity}

\section{Criterions}

\subsection{Feasibility}
\subsection{Optimality}

\section{Completeness}

\section{Probabilistic Completeness}

\section{Planning Under Uncertainty (Decision Theoretic Planning)}

All real life motion planning problems are faced with some level of uncertainty,
as errors can arise from a multiple of sources, whereas the most prevalent ones
are uncertainty in:
\begin{enumerate}
\item motion
\item sensing
\item the environment
\end{enumerate}
Thus the exact system state is never exactly known. Therefore planning under
uncertainty is done in a belief space. Which is a description of the state space
using probability distributions. (Partial observability?) Planning is done in
the information space, as opposed to the state-space. Uncertainties in
prediction and the current state exist. Forward and backward projections.

(probability constraint violation)

\subsection{A Game Against Nature}

Theta is nature's actions. U is the robot's action.

\subsection{A Model of Uncertainty}

If the discrete state space model is expanded to include \(\ Theta\) as the
space of nature actions, and \(\ theta_k\) is the action selected by nature at
step k, then a transition is modeled as
\[
  x_{k+1} = f(x_k,u_k,\theta_k)
\]
As the actions of nature are not available beforehand, that is - \(\theta_ k\)
is not given, thus we have
\[
  X_{k+1} = \set{ x_{k+1} \ in X \mid \exists\theta_k \in \Theta(x_k,u_k) \text{
      such that } x_{k+1} = f(x_k,u_k,\theta_k)}
\]
\cite{Lav06}.

\subsection{Discrete Planning with Nature}
a model of discrete planning with nature: \cite[pg,496]{Lav06}

A markov decision process is defined in \cite[pg.498]{Lav06} as:
\begin{enumerate}
\item A non empty \textit{state space} \(X\) which is a finite or countably
  finite set of \textit{states}.
\item For each state, \(x \in X\), a finite non-empty \textit{action space}
  \(U(x)\). It is assumed that \(U\) contains a special \textit{termination
    action}, which has the same effect as (cite the deterministic discrete
  planning model).
\item A finite non-empty \textit{nature action space} \(\Theta(x,u)\) for each
  \(x \in X\) and \(u \in U\).
\item A state transition function \(f\) that produces a state,
  \(f(x,u,\theta)\), for every \(x \in X\), \(u \in U\), and \(\theta \in
  \Theta\).
\item A set of \textit{stages}, each denoted by k, that begins at k=1 and
  continues indefinitely. Alternatively, there may be a fixed maximum stage \(k
  = K+1=F\).
\item An \textit{initial state} \(x \in X\). For some problems, this may not be
  specified, in which case a solution plan must be found from all initial
  states.

\item A \textit{goal set} \(X_G \subset X\).
\item A stage-additive cost functional L. Let \(\sim{\theta}_K\) denote the
  history of nature actions up to stage K. The cost functional may be applied to
  any combination of state, action, and nature histories to yield
  \[
    L(\sim{x}_F, \sim{u}_K, \sim{\theta}_K) = \sum_{k=1}^{K} l(x_k,u_k,\theta_k)
    + l_F(x_F)
  \]
  in which \(F = K + 1\). The termination action \(u_T\) is applied at some
  stage k, then for all \(i \geq k\), \(u_i = u_T\), \(x_i = x_k\), and
  \(l(x_i,u_T, \theta_i) = 0\).
\end{enumerate}
\cite[pg 498]{Lav06}

using this formulation either a feasible or optimal planning problem can be
formulated.

\subsection{Forward and Back-Projections (under uncertainty)}

\subsection{Separating a plan from its execution}

\subsection{Feedback}

\subsection{Information Space}
A stochastic environment. The information space is a general structure for
working with plans under conditions of uncertainty. Thus planning can be done
mostly as it is done in a state space, albeit in a higher dimension. Thus the
state transition function can be written as
\[
  \phi_{k+1} = f_{\mathcal{I}}\left( \phi_k, \mu_k, y_{k+1} \right)
\]

\subsection{Nondeterministic and Probabilistic Information Spaces}

\subsection{A Plan under Uncertainty}
Since it Expected-case analysis. Worst-case analysis.

Uncertainty comes from three sources in our model:
\begin{enumerate}
\item The Model (motion).
\item The observation (sensor).
\item The environment (map). Grid-based problems: Navigation: A goal position is
  to be reached, even though the map is unknown. The robot is allowed to solve
  the problem without fully exploring the environment. Only a part of the
  environment is needed to solve the problem. Searching: A goal state can only
  be identified when it is reached by a short-range sensor. The environment is
  systematically searched, but the search may terminate early if the goal is
  found.

  Problem-definition: Part of the problem is map-building, as the map is not
  provided exactly, and sensory map-information has to be used in order to
  successfully solve a given planning problem.
\end{enumerate}

\subsection{Visibility Polygon}
Let \(V(x)\) be the visibility polygon - that is the set of all points that are
visible from \(x\).

I-states enables us to solve problems without having a complete represenation of
the environment! (LAV06)chp12.2



\subsubsection{Decision Makers}

\subsection{Rewards?}

\section{Data Collection and Sensors}



\section{Optimality}



\section{Survey of Papers}

\subsection{Related Work}

http://msl.cs.uiuc.edu/~lavalle/cs497/jokane.pdf

http://robotics.cs.unc.edu/BeliefSpacePlanning/index.html

\section{Survey}

Following is a survey of motion planning techniques with a focus on their
application in unmanned ground vehichles (UGVs). First described is the relevant
sources of uncertainty in UGV motion planning, and then describes the relevant
techniques that have been applied to solve them in the literature. In general
uncertainty in UGV planning can be related to three sources:
\begin{enumerate}
\item Uncertainty in robot sensing.
\item Uncertainty in robot predicability.
\item Uncertainty in environment sensing.
\item Uncertainty in environment predicability.
\end{enumerate}
\cite{lavalleFrameworkMotionPlanning1995}

\subsection{Uncertainty in robot predicability}
Uncertainty in vehicle dynamics arises as the future robot configuration cannot
be predicted exactly. This results from modelling errors and/or limited
precision in the system's command tracking performance
\cite{dadkhahSurveyMotionPlanning2012}. Thus a transfer from one state to
another will not guarantee full knowledge of the vehichle in the next state.
This is referred to as \textit{automated sequential decision making} in the
literature, and the mathematical framework used to tackle such uncertainty is
the \textit{Markov Decision Process} (MDPs), used to formulate an optimal value
problem, which then can be solved for an optimal value function, and the
corresponding optimal policy \cite{Cassandra:1998:EAA:926710}. An introduction
to Markov decision processes can be found in \textit{grasping
  POMDs}\cite{kaelblingPlanningActingPartially1998}. This article uses apriori
known probability distributions in order to model uncertainty in the robot model
and in the sensors, then proceeds to use this to solve a number of different
objective functions. This probabilistic representation also gives enormous
savings on the computational power needed to solve a problem, by focusing on the
parts of space that are most likely to be encountered. The problem of solving
POMDs lies with the size of the state space when this solution strategy is
applied to solving real-world problems -- referred to as \textit{the curse of
  dimensionality}. In fact
\cite[Tsilkis]{christosh.papadimitriouComplexityMarkovDecision1987} showed that
solving such a problem is PSPACE-Complete and thus not tractable for real-life
applications. However approximate solutions are available, as shown by
\cite[Kaelbling et. al]{kaelblingPlanningActingPartially1998}. Therefore the
problem has to be solved by using techniques such as sampling the belief space,
as shown in \cite[Kearns]{kearnsSparseSamplingAlgorithm}. Another approach,
using Monte-Carlo simulation, is shown effective on large belief-spaces in
\cite{silverMonteCarloPlanningLarge}.

% \subsubsection{Optimal Control Based Approaches}

\subsection{Uncertainty in Environment Predicability}
If the robot has imperfect or non-existent a-priori maps, or noisy sensory data,
complete deterministic knowledge of the environment is impossible.

\subsubsection{Planning techniques for an uncertain environment}
For an UGV in an unknown environment, being able to have the situational
awareness to avoid collisions while adhering to the global planning
requirements, despite unpredicted obstacles appearing, is essential. Thus
environment sensing and mapping and re-planning in real-time is required. This
is referred to in the literature as planning in partially unknown environments.
One solution to this problem is through incremental graph-search algorithms, as
shown in \cite[Stentz]{stentzOptimalEfficientPath}, where the \textsl{D*}
algorithm is described as a method for optimal and efficient replanning in
partially unknown environments. Then \textit{Stentz}, a year later published the
\textit{Focused D*} algorithm a year later \cite{stentz1995focussed}, which
incorporates heuristics to reduce the total time taken for re-planning.
Still,\textit{D*} is computationally heavy, and algorithms have been created to
improve on the time-bound of \textit{D*}. One such implemplemenation is the
\textit{D*-lite} algorithm presented by \cite[Koenig]{koenig2002d}. These
incremental planners, which incorporates the previously calculated plan in the
solution of the newly arisen planning problem speeds up the planning cycles, but
can in many cases still not be enough for a viable real-time solution. Finding a
new plan within the allotted planning interval may simply not be possible, in
which case one can resort to \textit{Anytime} algorithms, like \cite[Karaman et.
al]{karamanAnytimeMotionPlanning2011}, which will find the approximation in the
given time interval. Anytime planners find a solution quickly, and then spends
the rest of the alloted time on improving it until time runs out. One such
example of both an anytime and incremental solution is
\cite[Likachev]{likhachevAnytimeSearchDynamic2008}. Another commonly used
sampling approach is the \textit{Rapidly Exploring Random Tree} (RRT), which has
shown itself useful in dealing with high dimensional state-spaces. Several
RRT-tweaks have been proposed over the years, where a good comparison of
different versions of RRT (RRT*, RRT*-smart), can be found in
\cite{noreenComparisonRRTRRT2016}. In \cite{melchiorParticleRRTPath2007} a
\textit{Particle-RRT} (pRRT) is an extension of the common RRT algorithm into
belief space, and then applied to a rover driving in rough terrain. By
propagating the uncertainty along the planned path, and running this procedure
multiple times, a cluster of nodes is formed. Nodes in the search tree are then
formed from these clusters, and a likelihood can then be assigned to each path
as the likelihood of succesfull execution can be quantified. Another
RRT-extension is given in \cite{Luders_2013}, which present the \textit{Closed
  Loop Rapidly exploring Random Tree} (CC-RRT) algorithm.


\subsubsection{Environment Mapping}
In the case that the map is only globally known, the local map may be wrong or
have errors, that the on-board local sensors will have to figure out. In order
to incorporate this into the planning procedure the field of mapping has to be
considered. In the literature this field is referred to as\textit{Simultaneous
  Location and Mapping} (SLAM). First of all virtually all robotic mapping
algorithms are probabilistic \cite{thrunRoboticMappingSurvey}. % TODO - read
% this survey
A method which is called \textit{Occupancy Grid} is presented in
\cite{elfes1989using}, in which the map is split up into cells, and each cell is
assigned a given probability of occupancy. It can incorporate information from
high-level maps using the same method that is used for estimating the occupancy
of a cell in a totally unknkown environment. As the model is based on Bayesian
estimation, the initial map-data is used ad apriori input to the probabilistic
model. Thus the usage of a map of the area is voluntary. In this way, unknown
cells (those that are not yet inspected by local sensors) can be assigned a high
probability of occupancy. An old article on how this can be implemented on a
real robot is given by \cite[Krugman]{kriegman1987mobile} More recent work on
the Occupancy Grid method can be found in
\cite{carrilloAutonomousRoboticExploration2015}, which explores the trade-off
between exploring a new area, and relying on the information already obtained in
order to solve the problem. A method to integrate current map-data with the
on-board filter information is given in \cite{gindeleBayesianOccupancyGrid2009}
using a Bayesian Occupancy Grid Filter for dynamic environments using prior map
knowledge.
\subsubsection{Integrating Planning and Mapping}
Not reasoning about the map and environment uncertainty can lead to crashes, as
further obstacles can be hidden behind other obstacles, and moving into
unexplored territory too fast will almost certainly lead to trouble. Using
Chance constrained programming as a solution to a problem modelled as a POMDP
vitus et. al \cite{vitusHierarchicalMethodStochastic2012} shows that
incorporating real-time sensing into the model, which updates the probability a
link in a graph will be traversable depending on how well it has been sensed.
Thus a method is developed for balancing exploration and apriori knowledge of
the environment. The method is also employed experimantally on a quad-copter for
proof of real-time applicability. A method for assigning different probabilities
to different paths is given in \cite{vandenbergLQGMPOptimizedPath2011} and is
based on the \textit{Linear Quadratic Gaussian Motion Planning} LQG-MP
algorithm. In \cite{kurniawatiGlobalMotionPlanning2012} a motion planner called
\textit{Guided Cluster Sampling} is used to that takes into account all three
sources of uncertainty for robots with active sensing capabilities. This method
builds on the POMDP framework by utilizing a more suitable sampling distribution
based on the observations done by the robots active sensors. In
\cite{yifenghuangRRTSLAMMotionPlanning2008} the \textit{RRT-SLAM} method is
introduced, where uncertainty is used in the RRT-planner by moving the state
space up a dimension, then this is joined together with a \textit{Simultaneous
  Location and Mapping} (SLAM) procedure. A ranking given on each path,
depending on its safety is given in
\cite{blakeEfficientComputationCollision2018}.
\cite{bryRapidlyexploringRandomBelief2011} gives an RRT algorithm that plans in
belief space, and incorporates this with an information region where the robot
has little uncertainty, and can thus localize itself. This can be a fine
extension to loss of localization on the robot, like a loss of gps for a certain
section of the map

\subsubsection{Reactive Planning} In case of adhering to the \textit{Plan
  Globally and Act Locally} paradigm, a reactive planner coupled with a global
path planner can provide a complete navigation solution, as shown in
\cite{djekouneSensorBasedNavigation2009}

