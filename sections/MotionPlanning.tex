\chapter{Motion Planning}
\label{sec:MotionPlanning}

\section{Overview}
Motion planning is the task of manipulating a robot's configurations so that,
given an initial and goal state or posture, the planner (black box), is able
to create a sequence of actions that gives a feasible or optimal path through
the overarching planning environment, usually referred to as the world space.
Thus a motion planner can be seen as a machine which given an the input: a
world, and inital and goal states, produces a sequence of actions to move the
robot from its initial to the desired goal configuration. This plan is then
passed on to the trajectory generator in the system. Generally, planners can be
separated into complete and non-complete planners, meaning that given enough
time, all motion planning problems are solvable, only the solution is NP-Hard.
Thus feasible solutions will have to make compromise, and a lot of planners in
use today are what is called probabilisticly complete, meaning that they
converge to a solution given infinite time. As such timeframes are of no use to
us, we generally have to settle for an approximation bound. There is a
difference between online and off-line motion planning, whereas the offline
algorithm plans in a static environment, the online algorithm is run
continuously. However an online algorithm can be simulated by running an offline
algorithm repeatedly for short intervals of time. However, this comes with the
drawback, that no guarantee can be made for completing a certain task.

\subsection{Building Blocks}

\subsection{The Robot and World Model}

The robot is/can be represented as...
model and transformation of a rigid body.

The world as we model it contains two kinds of objects

\begin{itemize}
\item \textbf{Robot:} Body that is modelled geometrically and is controllable via
  a motion plan.
\item \textbf{Obstacles:} Portions of the world that are occupied by something
  other than the robot.
\end{itemize}
\subsection{State Space}
\label{subsec:State}
Planning problems involve a state space. A state space is a set of all the
possible configurations a model can be in at any given time during the planning
work. The state space can be either discrete or continuous, and can be
represented either implicitly or explicitly. Which in our case of planning for a

The world model.

\subsubsection{Configuration Space}
\label{sec:configuration-space}
The configuration space is a general abstraction used as a model for a wide
variety of motion planning problems. It is a manifold that arise from the
transformations applied to the robot. If the robot has n-degrees of freedom, the
set of transformations is usually a manifold of dimension n. This manifold is
called the configuration space of the robot, and is shortened \(\\\mathcal{C}\).
Thus in order to solve a motion planning problem, a search must be performed in
the configuration space. Thus the motion planning problem is now made into a
question of finding the best path to traverse the given manifold. \cite{Lav06}.(characterise and describe the
configuration space's structure.)
In our case the configuration space can be modelled as a topological space of
the type \(SE(3)\), meaning special-Euclidean three.

Using generalised coordinates, the configuration of a robot can be modelled as a
vector with n variables for the position in the configuration space, in
literature this space is usually denoted by \(\mathcal{C}\). As is most common,
the robot is modelled as a rigid body, in which two bodies cannot overlap, thus
points in the configuration space is separated into two sets. One set for which
the robot configuration overlaps with another object in the configuration space,
and is called \(\mathcal{C}_{Obst}\), and given that the world space is
represented as \(\mathcal{W}\), the collision free space is the given as
,\(\mathcal{C}_{free} = \mathcal{W}\textbackslash \mathcal{C}_{Obst}\). The
obstacle space can be represented in a number of ways, but we will stick to
polygonal models of the obstacle space.

\subsubsection{Path and Trajectory}

\subsubsection{State Space}

\subsubsection{... Space}

Initial and Goal States

\subsection{Time}
\label{subsec:Time}

\subsection{Actions}
\label{subsec:Actions}

A plan is a sequence of actions that manipulate the state of the robot.

A planning problem involves a sequence of decisions made over time.


car-like model can look something like:

\section{Algorithms}
\subsection{Grid-based Search}
\label{subsec:gridbasedsearch}

\subsection{Interval-based Search}
\label{subsec:intervalbasedsearch}

\subsection{Geometric Algorithms}
\label{subsec:geometricalgorithms}

\subsection{Artificial Potential Fields}
\label{subsec:artificialpotentialfields}

\subsection{Sampling-based}
\label{subsec:samplingbased}









\section{Piano Mover's Problem}

\section{Discrete vs Continous}

\section{Sampling}

\section{State Space}

\section{Complexity}

\section{Criterions}

\subsection{Feasability}
\subsection{Optimality}

\section{Completeness}

\section{Probabilistic Completeness}

\section{Planning Under Uncertainty (Decision Theoretic Planning)}

\subsection{A Game Against Nature}

Theta is nature's actions. U is the robot's action.

\subsection{Information Space}
\subsubsection{Decision Makers}

\section{Optimality}


\section{Survey of Papers}

http://msl.cs.uiuc.edu/~lavalle/cs497/jokane.pdf

http://robotics.cs.unc.edu/BeliefSpacePlanning/index.html