\chapter{RRT}

The planning algorithm used in this thesis will build upon the \textit{Rapidly
  Exploring Tree} algorithm, which was first proposed by LaValle and . in TODO
\cite{TODO}.

\begin{figure}
  \centering
  \begin{subfigure}[b]{0.3\textwidth}
    \includegraphics[width=\textwidth]{plainRRT10}
    \caption{RRT-tree after 10 iterations.}
  \end{subfigure}
  \begin{subfigure}[b]{0.3\textwidth}
    \includegraphics[width=\textwidth]{plainRRT50}
    \caption{RRT-tree after 50 iterations.}
  \end{subfigure}
  \begin{subfigure}[b]{0.3\textwidth}
    \includegraphics[width=\textwidth]{plainRRT100}
    \caption{RRT-tree after 100 iterations.}
  \end{subfigure}
  \newline % Start the new line of plainRRT10.
  \begin{subfigure}[b]{0.3\textwidth}
    \includegraphics[width=\textwidth]{plainRRT500}
    \caption{RRT-tree after 500 iterations.}
  \end{subfigure}
  \begin{subfigure}[b]{0.3\textwidth}
    \includegraphics[width=\textwidth]{plainRRT1000}
    \caption{RRT-tree after 1000 iterations.}
  \end{subfigure}
  \begin{subfigure}[b]{0.3\textwidth}
    \includegraphics[width=\textwidth]{plainRRT10000}
    \caption{RRT-tree after 10000 iterations.}
  \end{subfigure}
\end{figure}

\subsection{General framework under differential constraints} (LaValle p.676.)


\subsection{RDT}


\subsection{Dynamic RRT}

\subsection{Funnel}

A Funnel is a motion primitive that is 'guaranteed' to take the vehicle from a
set of initial conditions to a set of goal states. 

\subsubsection{Composition of funnels}

In essence the 
Each funnel solves the subgoal of getting from the initial set of the funnel to
the goal set. Thus in essence each funnel solves the subproblem of getting from
one funnel to the next, and therefore composing funnels from some global initial
state to the goal state will have solve the motion planning problem with the
guarantees given by the tubes used for the planning task at hand.

TODO - insert pretty picture of funnel with the integral curves of the extremal
paths embedded in beautiful red.

\subsubsection{Reachability plot for the Ground-Vehicle}
TODO - plot the reachability for the ground-vehicle in some time interval using
simulations. 

TODO - plot A reachability tree in 3D using funnels - because different theta's
can exists at the same x,y positiions.

\subsection{Sampling}
It is important that the sampling sequence is dense in the space where sampling
occurs (state-space?), as we want resolution completeness in the end.
\subsubsection{How to obtain uniform sampling}
\subsubsection{How to define a good distance metric}
In general, it is not possible to get a perfect distance metric for our planning
problem, as this involves solving another optimal planning problem, and will
therefore be as, or more complex than the motion planning problem which is
already being solved. Therefore in general we will have to limit ourselves to
approximate distance metrics. The idea is to get as close to the optimal
cost-to-go function without having to compute expensive computations \cite{LaValle09}.
Distance metrics candidates in the RRT-Funnel algorithm:
\begin{itemize}
  \item Time - Since time can be found by simple summing up the time of all the
    funnels which need be added to get to the certain point in the configuration space.
  \item Lyapunov function - As the Lyapunov function can be seen as an energy
    function, the cost to go to a point can (probably) be used as a metric in
    the planning.
  \item Length of the shortest path between two configurations - ignoring collisions.
  \item A* search heuristics.
  \item Geometric - Stacking Funnels, where the shortest funnel of funnels wins.
    e.g. if the point is not in the cone projected out from the current state by
    the current motion primitives, pick the most extreme turn, and start over
    once again. If it can be reached by a turn, turn, then go straight for N-Funnels.
\end{itemize}

Note that most of these metrics are not metrics in the full sense, as they do
not fulfill the symmetric property, as under dynamic constraints going from a to
b, may not be the same as going from b to a. In most cases it is not true for a
nonholonomic vehicle.

\subsubsection{How to extend the tree?}

Should it be expanded into the dynamic Voronoi regions? If so, what are these regions?


\subsection{Motion Primitives}

A motion primitive is a discrete action chosen from some action set
\(\modelactionspace{}\), and applied as a constant action over a fixed period of
time. The time periods can vary in length, as the primitives do not need to have
the same time length. Thus in the model:

\begin{math}
  x_{k+1} = f_d(x_k,u_k)
\end{math} 

Where \(x_k = x((k-1)\delta{}t)\), and \(u_k\) is the action in
\(\modelactionspace{}_d\) that is applied from time \((k-1)\delta{}t\) to
\(k\delta{}t\). If we let \(\overline{u}^p\) be a motion primitive in
\(\modelactionspace{}\), which is a function from an interval of time, unto
\(modelactionspace{}\). Then by letting the interval of time start at 0 and stop
at \(t_F(\overline{u}^p)\), which has a final time that depends on the
particular prmitive \cite{LaValle09}.


\subsection{Reachability Graph}

\subsection{RRT-Funnels}


\subsubsection{Transforming and composing the tubes}

If we look at the cyclic coordinates of our model, we can see that using 'cyclic
coordinates', the dynamics is independent of position. Or rather:

\begin{math}
  \dot{\dot{q}} = \mathcal{L}(\dot{q},q)
\end{math}

TODO - finish this.


\subsubsection{Using Funnels as motion-primitives}

\subsubsection{Designing good motion primitives}