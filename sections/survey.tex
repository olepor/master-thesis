\chapter{Survey of Papers}

% \subsection{Related Work}

% http://msl.cs.uiuc.edu/~lavalle/cs497/jokane.pdf

% http://robotics.cs.unc.edu/BeliefSpacePlanning/index.html

\section{Focus}
Following is a survey of motion planning techniques incorporating uncertainty
with a focus on their application to unmanned ground vehicles (UGVs). First
described is the relevant sources of uncertainty in UGV motion planning, and
then the relevant techniques that have been applied to solve them in the
literature. In general uncertainty in UGV planning can be related to three
sources:
\begin{enumerate}
\item Uncertainty in predictability.
\item Uncertainty in environment.
\item Uncertainty in pose.
\end{enumerate}~\cite{lavalleFrameworkMotionPlanning1995}

\subsection{Uncertainty in predictability}
Uncertainty in vehicle dynamics arises as the future robot configuration cannot
be predicted exactly. This results from modeling errors and/or limited precision
in the system's command tracking
performance~\cite{dadkhahSurveyMotionPlanning2012}. Thus a transfer from one
state to another will not guarantee full knowledge of the vehicle in the next
state. This is referred to as \textit{automated sequential decision making} in
the literature, and the mathematical framework used to tackle such uncertainty
is the \textit{Markov Decision Process} (MDPs), used to formulate an optimal
value problem, which then can be solved for an optimal value function, and the
corresponding optimal policy~\cite{Cassandra:1998:EAA:926710}. An introduction
to Markov decision processes can be found in \textit{grasping
  POMDPs}\cite{kaelblingPlanningActingPartially1998}. This article uses apriori
known probability distributions in order to model uncertainty in the robot model
and in the sensors, then proceeds to use this to solve a number of different
objective functions. By focusing on the parts of space that are most likely to
be encountered, the problem is made tractable for real-life application. The
problem of solving POMDs lies with the size of the state space when this
solution strategy is applied to solving real-world problems - referred to as
\textit{the curse of dimensionality}. In fact
Tsilkis~\cite{christosh.papadimitriouComplexityMarkovDecision1987} showed that
solving such a problem is PSPACE-Complete and thus not tractable for real-life
applications. However approximate solutions are available, as shown by Kaelbling
et.\ al~\cite{kaelblingPlanningActingPartially1998}. Therefore the problem has
to be solved by using techniques such as sampling the belief space, as shown
in~\cite{kearnsSparseSamplingAlgorithm}. Another approach, using Monte-Carlo
simulation, is shown effective on large belief-spaces
in~\cite{silverMonteCarloPlanningLarge}.

% \subsubsection{Optimal Control Based Approaches}

\subsection{Uncertainty in Environment}
If the robot has imperfect or non-existent a-priori maps, or noisy sensory data,
complete deterministic knowledge of the environment is impossible. For an UGV in
an unknown environment, being able to have the situational awareness to avoid
collisions while adhering to the global planning requirements, despite
unpredicted obstacles appearing, is essential. Thus environment sensing and
mapping and re-planning in real-time is required. This is referred to in the
literature as planning in partially unknown environments. One solution to this
problem is through incremental graph-search algorithms, as shown
in~\cite{stentzOptimalEfficientPath}, where the \textsl{D*} algorithm is
described as a method for optimal and efficient re-planning in partially unknown
environments. Then \textit{Stentz}, a year later published the \textit{Focused
  D*} algorithm~\cite{stentz1995focussed}, which incorporates heuristics to
reduce the total time taken for re-planning. Still,\textit{D*} is
computationally heavy, and algorithms have been created to improve on the
time-bound of \textit{D*}. One such implementation is the \textit{D*-lite}
algorithm presented by Koenig~\cite{koenig2002d}. These incremental planners,
which incorporates the previously calculated plan in the solution of the newly
arisen planning problem speeds up the planning cycles, but can in many cases
still not be enough for a viable real-time solution. Finding a new plan within
the allotted planning interval may simply not be possible, in which case one can
resort to \textit{Anytime} algorithms, like Karaman et.\
al~\cite{karamanAnytimeMotionPlanning2011}, which will find an approximation in
the given time interval. Anytime planners find a solution quickly, and then
spends the rest of the allotted time on improving it until time runs out. One
such example of both an anytime and incremental solution is given
in~\cite{likhachevAnytimeSearchDynamic2008}. Another commonly used technique in
dealing with large state-spaces is sampling, and a popular approach to sampling,
is the \textit{Rapidly Exploring Random Tree} (RRT), which has shown itself
useful in dealing with high dimensional state-spaces. Several RRT-tweaks and
etensions have been proposed over the years, where a good comparison of
different versions of RRT (RRT*, RRT*-smart), can be found
in~\cite{noreenComparisonRRTRRT2016}. In order to deal with uncertainty, the
\textit{Particle-RRT} (pRRT) is proposed as an extension of the common RRT
algorithm into belief space, and then applied to a rover driving in rough
terrain~\cite{melchiorParticleRRTPath2007}. By propagating the uncertainty along
the planned path, and running this procedure multiple times, a cluster of nodes
is formed. Nodes in the search tree are then formed from these clusters, and a
likelihood can then be assigned to each path. Another RRT-extension is given
in~\cite{Luders_2013}, which present the \textit{Closed Loop Rapidly exploring
  Random Tree} (CC-RRT) algorithm, which can be employed for efficient
identification and execution of probabilistically safe paths both off-line and
on-line in an unknown and uncertain environment.

\subsubsection{Environment Mapping}
In the case that the map is globally known, the local map may be wrong or have
errors that the on-board local sensors will have to figure out. In the
literature this field is referred to as \textit{Simultaneous Location and
  Mapping} (SLAM). First of all virtually all robotic mapping algorithms are
probabilistic~\cite{thrunRoboticMappingSurvey}. % TODO - read
% this survey
A method which is called \textit{Occupancy Grid} is presented
in~\cite{elfes1989using}, in which the map is split up into cells, and each cell
is assigned a given probability of occupancy. It can incorporate information
from high-level maps using the same method that is used for estimating the
occupancy of a cell in a totally unknown environment. As the model is based on
Bayesian estimation, the initial map-data is used as apriori input to the
probabilistic model. Thus the usage of a map of the area is voluntary. In this
way, unknown cells (those that are not yet inspected by local sensors) can be
assigned a high probability of occupancy. An old article on how this can be
implemented on a real robot is given by Kriegman~\cite{kriegman1987mobile} More
recent work on the Occupancy Grid method can be found
in~\cite{carrilloAutonomousRoboticExploration2015}, which explores the trade-off
between exploring a new area, and relying on the information already obtained in
order to solve the problem. A method to integrate current map-data with the
on-board filter information is given in~\cite{gindeleBayesianOccupancyGrid2009}
using a Bayesian Occupancy Grid Filter for dynamic environments using prior map
knowledge.

\subsubsection{Integrating Planning and Mapping}
Not reasoning about the map and environment uncertainty can lead to crashes, as
further obstacles can be hidden behind known obstacles, and moving into
unexplored territory too fast will almost certainly lead to trouble. Thus using
a combination of sensing, mapping and re-planning is employed in part of the
literature on autonomous motion planning for UGVs. One method of integrating
mapping and planning is \textit{Chance constrained programming } as shown by
Vitus et.\ al~\cite{vitusHierarchicalMethodStochastic2012}. Here it is shown
that incorporating real-time sensing into the model, which updates the
probability a link in a graph will be traversable depending on how well it has
been sensed, is beneficial to succesful operation in an unknown environment.
Then a method is developed for balancing exploration and apriori knowledge of
the environment, in the face of uncertainties in the map. The method is also
employed experimentally on a quad-copter for proof of real-time applicability.
Assigning different probabilities to different paths in the face of uncertainty,
is an interesting way of dealing with e.g. time constraints, such as choosing a
faster path, with higher uncertainty when time is short. A method for doing so
is given in~\cite{vandenbergLQGMPOptimizedPath2011} and is based on the
\textit{Linear Quadratic Gaussian Motion Planning} LQG-MP algorithm. Another
more recent implementation is~\cite{blakeEfficientComputationCollision2018}.
In~\cite{kurniawatiGlobalMotionPlanning2012} a motion planner called
\textit{Guided Cluster Sampling} is used to that takes into account all three
sources of uncertainty for robots with active sensing capabilities. This method
builds on the POMDP framework by utilizing a more suitable sampling distribution
based on the observations done by the robots active sensors.
In~\cite{yifenghuangRRTSLAMMotionPlanning2008} the \textit{RRT-SLAM} method is
introduced, where uncertainty is used in the RRT-planner by moving the state
space up a dimension, then this is joined together with a \textit{Simultaneous
  Location and Mapping} (SLAM) procedure.
\cite{bryRapidlyexploringRandomBelief2011} gives an RRT algorithm that plans in
belief space, and incorporates this with an information region where the robot
has little uncertainty, and can thus localize itself. An information region is a
part of the map where the uncertainty in the environment is low, and the robot
can thus relocate itself. Another approach to dealing with uncertainty and
mapping is the \textit{reactive planners}. In case of adhering to the
\textit{Plan Globally and Act Locally} paradigm, a reactive planner coupled with
a global path planner can provide a complete navigation solution, as shown
in~\cite{djekouneSensorBasedNavigation2009}. This is useful in that a perfect
map of the environment may not be available. Worse even, the environment may
have changed, as nature is not static. Trees fall over, and floods move ground,
and create hindrances that might not be seen from a map. Not knowing where a
hindrance might lie, or whether one is located outside the sight of the onboard
sensors.

\section{Uncertainty in Pose}
Although our UGV incorporates GPS technology, it's accuracy in highly wooden
areas can be poor, if not non-existent, and thus lead to high uncertainty in the
robot's localization with respect to the map of the environment. Then the robot
has to localize itself using the remaining sensors aboard. This leads once again
to the POMDP model, which we already know has a very high complexity. Navigating
a robot with uncertain pose, in rough terrain is investigated
in~\cite{haitMotionPlanningRough1996}, where robustness in control is ensured by
imposing the validity of a configuration domain around the planned trajectory,
through the notion of a safe neighbourhood. Here a configuration is only
considered safe, if all it's neighbourhood configurations are also safe. It also
links position uncertainty with uncertainty in the map elevation model. Another
paper building on the traversal of rough terrain method given previously
is~\cite{iagnemmaRapidPhysicsbasedRoughterrain1999}. It also gives a nice
overview of how to incorporate uncertainty into the path-planning algorithm.

% TODO - implement automatic update of table-1 similar to that presented in:https://ieeexplore.ieee.org/stamp/stamp.jsp?tp=&arnumber=7339478

\section{Implementations}
A quick survey of implementations of motion planners on UGVs operating in
terrain is \cite{kellyReliableRoadAutonomous2006}, where the implementation of a
hierarchichal motion planner is shown.