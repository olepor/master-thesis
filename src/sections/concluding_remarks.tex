\chapter{Concluding Remarks}
\label{chp:concluding-remarks}

\section{Summary}

This thesis has developed the \rrtfunnel{} motion planning algorithm for
providing robust motion planning in known environments for nonlinear dynamical
systems with bounded uncertainty. The planner solved these problems through
isolating the uncertainty for a nonlinear system into provably \textit{robust
  motion primitives} that are guaranteed to provide safe traversal through the
overarching planning environment with the guarantees provided by the \ac{SOS}
programming framework. These motion primitives were then employed by a global
\ac{RRT} motion planner in order to compose a safe path through the planning
environment.

\section{Conclusion}

In conclusion the reachable sets calculated through the \ac{SOS} programming
framework was shown to provide safe traversal through the parametrization of an
outer approximation for the reachable set of the nonlinear dynamical system at
hand. The algorithm was verified through simulations in a virtual forest for a
simple airplane model. These experiments showed that the \rrtfunnel{} algorithm
is significantly more robust than an algorithm which does not handle
uncertainties explicitly, but employs some conservative heuristics, such as
maximizing the distance to the closest obstacle in order to avoid a collision
during execution of the plan. In fact, the \rrtfunnel{} algorithm did not
collide once.


\section{Future Work}

\subsection{The Global Motion Planner}

The current implementation of the \rrtfunnel{} algorithm relies on the basic
\ac{RRT} motion planning algorithm as the global planner. This is a simple
motion planning algorithm however, and there exists a number of extensions that
can be applied. The lowest hanging fruit will probably be making the algorithm
bi-directional. Also, any other discrete global motion planner can be applied,
such as A*, D*, etc.

\subsection{Exploit System Symmetries}

The funnel library could be made sparse by exploiting symmetries in the
dynamical model employed. As an example, there is really nothing separating a
left turn from a right turn, yet they are different motion primitives in the
funnel library of this thesis.

\subsection{Dynamically Shrink the Funnels}

If the funnels handle no uncertainty, every sub-level set of a funnel is a
funnel, following from the definition of the Lyapunov function. As such, the whole funnel can be shrunk to fit inside smaller
openings. Research to what extent this is possible with uncertain funnels, as in
this case, every sub-level set is not necessarily invariant.

\subsection{Move the Funnel Generation On-Line}

For now, the computational complexity of generating funnels on-line is too high
with the current computational resources, and the formulations applied. However,
there are advances made, such as \textit{SDOS} and \textit{DSOS} formulations
available of \ac{SOS} optimization problems~\cite{ahmadi2014dsos}. This could
for lower dimensional systems, enable an on-line hybrid implementation where
funnels are generated, and stored as they have been generated and then reused.




