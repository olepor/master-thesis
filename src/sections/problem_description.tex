\section{Problem Description}

Current motion planning algorithms do not in the general case handle the problem
of uncertainty during planning. This is not optimal, in that all real-world
motion planning problems have to handle uncertainties. Therefore, this thesis
focus is on developing a motion planning algorithm that is robust to
uncertainty, and verifying the safe traversal using the algorithm through
simulation.
% \subsection{What has been done by others}

% The two main parts of the \rrtfunnel{} algorithm is the funnel generation
% framework through \ac{SOS} programming. Although this is used in a lot of
% different research, the strand that this thesis is based upon is a series of
% papers~\cite{Tobenkin_2011,tedrakeLQRtreesFeedbackMotion2009,majumdarRobustOnlineMotion2013,majumdarFunnelLibrariesRealtime2017,ahmadi2014dsos}
% with the main focus being on \cite{majumdarFunnelLibrariesRealtime2017}. The
% second part is based on~\textcite{article} and the \ac{RRT} motion planning
% algorithm, which is a simple algorithm that seeks to build a random tree that
% quickly expands deeply into the state space, and due to its random nature is
% good at avoiding local extrema.

\subsection{Scientific contributions of this thesis}

The contribution of this thesis is the combination of the \ac{RRT} motion
planning algorithm with the guaranteed safe traversal provided by the \ac{SOS}
programming framework. The \textit{finite time reachable sets} produced by the
\ac{SOS} program are then employed as robust motion primitives to the \ac{RRT}
motion planner, letting it create a tree of safe maneuvers as it searches for a
path to the goal, all the while guaranteeing safe traversal for a nonlinear
dynamical model -- even in the face of uncertainty.


\section{Outline}

The rest of the thesis is organized as follows:
\begin{description}
\item[\cref{chp:survey-of-papers}] provides a survey of current motion planning
  research focused on handling uncertainty in the environment, the system state
  and the surrounding environment, as well as some more in depth on the funnel
  theory employed in this thesis.
    
\item[\cref{chp:preliminaries}] provides some introductory theory, first to
  motion planning as a general topic, then introduces the funnel generation
  theory through \ac{SOS} programming, and then develops the basic theory needed
  for understanding the inner workings of the \ac{RRT} algorithm.
    
\item[\cref{chp:method}] develops the \rrtfunnel{} algorithm through two parts.
  Firstly it develops robust motion primitives through the \ac{SOS} programming
  framework. Then it develops the needed heuristics and sampling distribution
  for the \ac{RRT} motion planner and then incorporates the funnels from the
  first part as the extension operators to the tree that the algorithm builds
  through the configuration space.
    
\item[\cref{chp:experiments}] contains the description of how to create and
  setup the experiment environment and the benchmark planner that is used in
  comparing the performance of the \rrtfunnel{} algorithm. The results follow
  right after.
    
\item[\cref{chp:discussion}] gives a discussion of the results and a pointer to
  further work on the problem.

\item[\cref{sec:first-app}] gives a basic introduction to the theory of \ac{SOS}
  verification of dynamical systems.

\item[\cref{AppendixB}] contains some code examples from the code for the
  implementation of the \rrtfunnel{} algorithm. The rest of which can be found
  in~\cite{MasterThesisCode2019}.

\end{description}


