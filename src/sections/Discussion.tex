\chapter{Discussion}

Although the motion primitives of the \rrtfunnel{} algorithm makes for a smooth
and robust traversal of the state-space, it is not in general probabilisticly
complete, as there are configurations that it cannot reach due to the discrete
nature of the motion primitives. This problem can be, and is solved
in~\cite{vonasekGlobalMotionPlanning2013}, through the addition of a randomly
sampled control input, in addition to the randomly sampled motion primitives.
This approach was not exploited for the \rrtfunnel{} algorithm as it would
remove the robustness guarantee that come with the funnel motion primitives.

The algorithm does take into account uncertainty in both pose and
predictability, but not the surrounding environment.

The algorithm takes only kinematic constraints, and as such is severly limited
in that some plans might not be executable on an actual vehicle which is subject
to actuator constraints, forces and torques. Actuator constraints could be
enveloped in the current implementation
however~\cite{majumdarFunnelLibrariesRealtime2017}.

The implementation in this thesis leveraged the discrete sampling of points in
order to generate funnels around these discrete points. However, a continuous
approach might have been better suited, as the time for generating the funnels
off-line are not important for the algorithm at runtime. It did save a lot of
time in prototyping though.

The algorithm could in theory leverage the symmetries in the dynamics, and hold
a much sparser funnel library, and then simply mirror them at runtime, then the
mirror of another funnel is needed - say a left turn instead of a right, but
this implementation has to be considered more as a proof of concept than a
fine-tuned lean and mean implementation ready for use of a proper vehicle.

