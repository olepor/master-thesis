\chapter{Discussion}
\label{chp:discussion}

The strength of the algorithm is that through separating the uncertainty
handling into an off-line pre-computation phase. Therefore, the global motion
planner does not need to be significantly more complex than if it had not
reasoned about uncertainty. To some degree it can remain completely oblivious to
the overarching problem difficulty of uncertainties for a complex nonlinear
system. In fact once the motion primitives have been calculated and verified
off-line, they might as well be employed in any global motion planner able to
handle discrete motion primitives.


Note that the funnel is an outer reachable set for the problem at hand, which
means that it is an approximation. Therefore, since the Lyapunov function
employed is quadratic it will always be symmetric around the trajector verified.
This means that even though the real nonlinear system dynamics can have a tight
reachable set on one side of the trajector, the symmetry of the quadratic
Lyapunov function might lead the funnel to be too conservative on one side of
the trajector. Also, the verified funnels are verified discretely and hence are
only valid up to the numeric tolerances of the platform it is computed on, and
the precision of the solver employed.


The algorithm was verified through simulations in a virtual forest for a simple
airplane model. These experiments showed that the \rrtfunnel{} algorithm is
significantly more robust than an algorithm which does not handle uncertainties
explicitly, but employs some conservative heuristics, such as maximizing the
distance to the closest obstacle in order to avoid a collision during execution
of the plan.


Although the \rrtfunnel{} motion planning algorithm does provide safe traversal
of the configuration space, there are some downsides to the algorithm that
should be noted/improved:
\begin{itemize}

\item Even though the choice of using motion primitives for the \rrtfunnel{}
  algorithm makes for a smooth and robust traversal of the state-space, it is
  not in general probabilisticly complete, as there are configurations that it
  cannot reach due to the discrete nature of the motion primitives. This problem
  can be, and is solved by \textcite{vonasekGlobalMotionPlanning2013}, through
  the addition of a randomly sampled control input, in addition to the randomly
  sampled motion primitives. This approach was not exploited for the
  \rrtfunnel{} algorithm as it would remove the robustness guarantee that come
  with the funnel motion primitives.

\item The algorithm takes only kinematic constraints, and as such is limited in
  that some plans might not be executable on an actual airplane which is subject
  to actuator constraints, forces and torques. Actuator constraints could be
  enveloped in the current implementation
  (see~\cite[sec.4.3.3]{majumdarFunnelLibrariesRealtime2017}).

\item The implementation in this thesis leveraged the discrete sampling of
  points in order to generate funnels around these discrete points. However, a
  continuous approach might have been better suited, as the time for generating
  the funnels off-line are not important for the algorithm at runtime.

\end{itemize}

There are multiple ways of improving the algorithm at hand available, an outline
of a few are:

\begin{itemize}

  \item In general not every sub-level set of the uncertain funnel is invariant,
  however some investigations into this could significantly reduce the size of
  the funnel through choosing an invariant sub-level set of the funnel in the
  cases where a passage is narrow.

\item The algorithm could exploit symmetries in the dynamics, and simply mirror
  a motion primitive if needed. Hence a sparser motion primitive library could
  be employed.

\end{itemize}



