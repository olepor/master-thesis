\chapter{Discussion}
\label{chp:discussion}

In this thesis it is shown that the formulation of the search for a Lyapunov
function to parameterize the reachable set for an uncertain dynamical system
does indeed provide a tight outer approximation of the true reachable set for
the system. This reachable set surrounding a basis of trajectories is referred
to as funnels, and does provide safe traversal from the initial condition set of
the funnel to the outlet. These funnels are then used as motion primitives by
the global \ac{RRT} motion planner. The combination of the reachable sets as
robust motion primitives with a discrete \ac{RRT} motion planner is what is
referred to as the \rrtfunnel{} algorithm.

The strength of the algorithm lies in that it can separate handling of the
uncertainty into an off-line pre-computation phase. Therefore, the global motion
planner does not need to be significantly more complex than if it had not
reasoned about uncertainty. It can remain completely oblivious to
the overarching problem difficulty of uncertainties for a complex nonlinear
system. In fact once the motion primitives have been calculated and verified
off-line, they might as well be employed in any global motion planner able to
handle discrete motion primitives.

The funnels were shown through Monte-Carlo simulation to provide true outer
approximations of the reachable set for the dynamical system. Note that since
the Lyapunov function employed is quadratic it will always be symmetric around
the trajector verified. This means that even though the real nonlinear system
dynamics can have a tight reachable set on one side of the trajector, the
symmetry of the quadratic Lyapunov function might lead the funnel to be too
conservative on one side of the trajector. Also, the verified funnels are
verified discretely and hence are only valid up to the numeric tolerances of the
platform it is computed on, and the precision of the solver employed. Still, the
funnels generated can be said to be tight.

Even though the robustness guarantees were lost in the experiments section, due
to the planner not handling multiple controller inputs, the \rrtfunnel{}
algorithm did not cause the airplane to collide a single time over the run of
\(300\) simulation runs. This was in starch contrast to the benchmark planner,
which did not handle uncertainty at all, and instead relied on avoiding the
obstacles by as big a margin as possible. Therefore, not only was the
\rrtfunnel{} planner able to provide significantly safer traversal of the
experiment environment, but it could also provide more direct traversal, as the
distance to the trees did not matter.

The algorithm does however have some potential for improvement. The choice of
using motion primitives calculated off-line will in general not make the planner
probabilisticly complete, and as such, one is not guaranteed to find a solution
given infinite time. A guarantee that regular \ac{RRT}'s do provide.

The algorithm takes only kinematic constraints into account, and as such is
limited in that some plans might not be executable on an actual airplane, or a
model which are both subject to actuator constraints, forces and torques.
However, these issues could be solved through the addition of actuator
constraints (see~\cite[sec.4.3.3]{majumdarFunnelLibrariesRealtime2017}).

The implementation of the algorithm in this thesis leveraged the discrete
sampling of points in order to generate funnels around these discrete points.
However, a continuous approach might have been better suited, as the time for
generating the funnels off-line are not important for the algorithm at runtime.

