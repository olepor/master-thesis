\chapter{Discussion}
\label{chp:discussion}

Although the motion primitives of the \rrtfunnel{} algorithm makes for a smooth
and robust traversal of the state-space, it is not in general probabilisticly
complete, as there are configurations that it cannot reach due to the discrete
nature of the motion primitives. This problem can be, and is solved
in~\cite{vonasekGlobalMotionPlanning2013}, through the addition of a randomly
sampled control input, in addition to the randomly sampled motion primitives.
This approach was not exploited for the \rrtfunnel{} algorithm as it would
remove the robustness guarantee that come with the funnel motion primitives.

The algorithm does take into account uncertainty in both pose and
predictability, but not the surrounding environment.

The algorithm takes only kinematic constraints, and as such is severly limited
in that some plans might not be executable on an actual vehicle which is subject
to actuator constraints, forces and torques. Actuator constraints could be
enveloped in the current implementation
however~\cite{majumdarFunnelLibrariesRealtime2017}.

The implementation in this thesis leveraged the discrete sampling of points in
order to generate funnels around these discrete points. However, a continuous
approach might have been better suited, as the time for generating the funnels
off-line are not important for the algorithm at runtime. It did save a lot of
time in prototyping though.

The algorithm could in theory leverage the symmetries in the dynamics, and hold
a much sparser funnel library, and then simply mirror them at runtime, then the
mirror of another funnel is needed - say a left turn instead of a right, but
this implementation has to be considered more as a proof of concept than a
fine-tuned lean and mean implementation ready for use of a proper vehicle.

In general not every sub-level set of the uncertain funnel is invariant, however
some investigations into this could significantly reduce the size of the funnel
through choosing an invariant sub-level set of the funnel in the cases where a
passage is narrow.

The difficulty of the traversed obstacle course could be significantly increased
with a few modifications to the \rrtfunnel{} algorithm. For one, every time
sample can be a node in the graph, and not simply the inlet's and outlet's of
the funnels. This would increase the resolution at which the algorithm operates
significantly, hence enabling the difficulty to be raised.

The verified funnels are verified discretely and hence are only valid up to the
numeric tolerances of the platform it is computed on, and the precision of the
solver employed.

The reachable set is an outer approximation for the problem at hand, and since
the Lyapunov function employed is quadratic it will always be symmetric around
an axis for the system. This means that even though the system can be pretty
tight on one side of a trajector, the symmetry might lead the set to be way too
big on the other, as a result of the quadratics involved.

The \rrtfunnel{} algorithm does provide safe guaranteed passage through an
environment, however it does not offer any optimality guarantees involved with
the traversal. This might be an interesting topic for further research however.