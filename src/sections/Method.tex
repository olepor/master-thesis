\chapter{Method}

This chapter will introduce and develop the \rrtfunnel\ algorithm, through two
means: Developing robust motion primitives through the \ac{SOS} programming
framework, based on the work in \cite{majumdarFunnelLibrariesRealtime2017}, and
second deploy these funnels as motion primitives in a discrete \ac{RRT} planner,
based on \cite{lavalleLav98cPdf}.

\section{Defining funnels}

The funnel computations will be based on the \ac{SOS} theory as developed in
\ref{sec:Funnels}. Given a trajectory, the goal is to compute a robust invariant
set around the trajectory that will 'guarantee' that the planner is free from
collisions during execution of the obtained motion plan. This robustly invariant
set is parameterized through Lyapunov function candidates, that, in this case,
will be based upon the \ac{PSD} matrix that the time-invariant \ac{LQR}
controller produces. The following presentations will be based on
\cite{tobenkinInvariantFunnelsTrajectories2010,
  tedrakeLQRtreesFeedbackMotion2009, majumdarRobustOnlineMotion2013} , but
mainly follow the formulations, and syntax from
\cite{majumdarFunnelLibrariesRealtime2017}.

Thus, given the nonlinear dynamical system
\begin{equation}
  \label{eq:dynamicalsystem}
  \dot{x} = f(x(t), u(t))
\end{equation}
with \(x(t)\) the state of the system at time \(t\) and \(u(t)\) the control
input. Assume that a open loop nominal trajectory \(x_0 \colon [0,T] \rightarrow
\R^n\) with control input \(u_0 \colon [0,T] \rightarrow \R^n\) is given, and
define a change of coordinates into the error coordinate frame
\[
  \bar{x}(t) = (x - x_0)(t) \\
  \bar{u}(t) = (u - u_0)(t).
\]
Then, changing \ref{eq:dynamicalsystem} to these new coordinates one obtaines
\begin{equation}
  \dot{\bar{x}} = \dot{x} - \dot{x}_0 = f(x_0(t) + \bar{x}(t), u_0(t) + \bar{u}(t)) - \dot{x}_0(t)
\end{equation}

In order to compute a parameterized reachable set through \ac{SOS} programming
the system \ref{eq:dynamicalsystem} needs to be polynomial, and parameterized by
\(x\) and \(t\), since the trajectory is parametrized by \(x\) and \(t\).
Therefore, through the use of a \ac{TV-LQR} controller, the control input can be
eliminated from the dynamical equation, giving
\begin{equation}
  \label{eq:dynamicclosedloop}
  \dot{\bar{x}} = f_{cl}(t,\bar{x}(t)).
\end{equation}
However, the dynamical system may still not be polynomial, which is a necessary
condition in order for this to be verified using \ac{SOS} programming. Expanding
the system \ref{eq:dynamicclosedloop} around the nominal trajectory \(x_0\)
through a Taylor expansion of some degree high enough to capture the
nonlinearities of the system.

The goal is to parametrize a tight outer approximation of the set of states the
system may transition into during the time interval \([0,T]\). Given that
\(F(t)\) is the set of states the system (\ref{eq:dynamicclosedloop}) can be in
at time \(t\), then
\begin{equation}
  \label{eq:reachableset}
  \bar{x}(0) \in \mathcal{X}_0 \implies \x(t) \in F(t), \, \forall t \in [0,T]
\end{equation} \cite{majumdarFunnelLibrariesRealtime2017} 
where \(\mathcal{X}_0\) is the initial condition set, and \(F(t) \subset \R^n\).

A funnel is defined in \cite{majumdarFunnelLibrariesRealtime2017} as
\begin{definition}
  \label{def:funnel}
  A funnel associatied with a closed-loop dynamical system \(\dot{\bar{x}} =
  f_{cl}(t,x(t))\) is a map \(F \colon [0,T] \rightarrow \mathcal{P}(\R^n)\),
  from the time interval \([0,T]\) to the power set (\ie the set of subsets) of
  \(\R^n\) so that the sets \(F(t)\) satisfy the condition.
  \ref{eq:reachableset}.
\end{definition}
Thus, \(F(t)\) is the set of reachable states that the system can be in at time
\(t\).

Next, the reachable set is paramterized through the use of Lyapunov functions,
which yields
\begin{equation}
  F(t) = \set{\bar{x}(t) \mid V(t, \bar{x}(t) \leq \rho (t))}
\end{equation}
where \(\rho (t) \colon [0,T] \rightarrow \R^+\), is a function which limits the
size of the reachable set, and \(V(t,\bar{x}(t))\) is a Lyapunov function \(V
\colon [0,T] \times \R^n \rightarrow \R^+\).

Then, by setting \(\mathcal{X}_0 \subset F(0,\bar{x})\), one can derive the
sufficient condition for containing the reachable \ref{eq:reachableset} set in
the Lyapunov function paramterization
\begin{equation}
  V(t,\bar{x}) = \rho(t) \implies \dot{V}(t,\bar{x}) < \dot{rho}(t), \, \forall t \in [0,T]
\end{equation}
with \(\dot{V}(t,\bar{x})\) computed as
\begin{equation}
  \label{eq:funnelsufficient}
  \dot{V}(t,\bar{x}) = \frac{\partial V(t,\bar{x})}{\partial x} f_{cl}(t,\bar{x}) + \frac{\partial V(t,\bar{x})}{\partial t}
\end{equation}

Currently there are no limitations on the functions \(V\) and \(\rho\), and
hence there exists infinitely many functions with different sized reachable sets
that satisfies \ref{eq:funnelsifficient}, and is a valid funnel in the sense of
definition \ref{def:funnel}. In order for efficient planning to take place, the
motion primitives, meaning the size of the funnels, should be as small as
possible, and it is therefore that the size of the funnels is minimized using
the following optimization problem \cite{majumdarFunnelLibrariesRealtime2017}

\begin{align}
  &\underset{V,\rho}{\text{inf}} \; &&\int_{0}^{T} vol(F(t)) dt  \\
  &\text{subject to} && V(t,\bar{x}) = \rho (t) \implies \dot{V}(t,\bar{x}) < \rho (t), \, \forall t \in [0,T] \\
  && &\mathcal{X}_0 \subset F(0,\bar{x})
\end{align} 


\section{Computing funnels}

