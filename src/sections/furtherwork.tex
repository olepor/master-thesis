\chapter{Further work}

Following is a collection of suggestions for further extensions and improvements
to the \rrtfunnel{} motion planning algorithm.

\begin{itemize}

\item The funnels computations can be made continous in order to regain a
  tighter approximation of the funnel. The computational time, altough it would
  grow significanlty, is not of concern as long as the funnels are computed
  off-line.

\item Increase the sophistication of the \ac{RRT} algorithm\ie through growing a
  bi-directional planning tree.
  
\item Any other discrete motion planning technique can be employed for the
  global planner, such as A*, D* etc.

\item Extend the algorithm to exploit the symmetry of the dynamical system at
  hand. As an example, there is nothing really seperating a left turn from a
  right turn, yet they are different motion primitives in the basis set.

\item At least in the case of funnels without uncertainty, every sub-level set
  of a funnel is a funnel, and as such, the whole funnel can be shrunk to fit
  inside smaller openings. Research to what extent this is possible with
  uncertain funnels, as in this case, every sub-level set is not necessarily
  invariant.

  \item An on-line hybrid implementation where funnels are generated, and stored
    as they have been generated and then reused would be interesting.

  \item  Investigate what is the smartest way of choosing endpoints for the
    funnels generated. What structure makes the most sense? Covers the most
    space? etc.

  \item The funnels can be locally shifted in the case of a collision, as the
    subset test may allow for some wiggle-room.

  \item The RRT algorithm can be expanded to handle anytime (re-planning) in the
    original graph.

  \item It would be interesting to see the funnels overlaid the optimal Dubin's paths.

  \item Maybe implement some sort of informed sampling.

  \item Custom sampling heuristics, like only sampling in the proximity of the
    basin of attraction of the tree.

\end{itemize}