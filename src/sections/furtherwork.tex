\chapter{Further work}

\begin{itemize}
  \item The funnels can be made continous once again, in order to regain
    exactness. The computational time is not really of huge concern when the
    funnels are computed off-line, and huge computational resources can be
    thrown at the problem once the development phase is done.
  \item The RRT algorithm can be modified in any number of ways.
    * Grow it backwards from the goal, instead of from the origin.
    * This can further be exanded for 

  \item Try any other discrete motion planning technique, such as Grid-based
    planner. Any of the heuristic planners such as A*, D* etc.

  \item There is plenty of symmetry that goes unused in the thesis. As an
    example, there is nothing really seperating a left turn from a right turn,
    yet they are different motion primitives in the basis set.

  \item Bi-Directional planning tree.

  \item At least in the case of funnels without uncertainty, every sub-level set
    of a funnel is a funnel, and as such, the whole funnel can be shrunk to fit
    inside smaller openings. Research to what extent this is possible with
    uncertain funnels.

  \item An on-line hybrid implementation where funnels are generated, and stored
    as they have been generated and then reused would be interesting.

  \item  Investigate what is the smartest way of choosing endpoints for the
    funnels generated. What structure makes the most sense? Covers the most
    space? etc.

  \item The funnels can be locally shifted in the case of a collision, as the
    subset test may allow for some wiggle-room.

  \item The RRT algorithm could probably be expanded to handle anytime
    (re-planning) in the original graph.

  \item It would be interesting to see the funnels overlaid the optimal Dubin's paths.

  \item Maybe implement some sort of informed sampling.

  \item Add an abort maneuver to the \rrtfunnel{} algorithm, so that if the
    system leaves the funnel, execute a loiter or abort maneuver. Eg. for a
    vehicle. Stop!

\end{itemize}