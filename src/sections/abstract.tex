% \chapter{Abstract}
% OPTIONAL SOLUTION: Use these settings if you are writing a monograph and have
% only one abstract. Do not use them if you have an abstract at the beginning of
% each chapter/paper.

\abstractintoc{} % Add abstract to Table of Contents
\abstractnum % Format abstract like a chapter

\begin{abstract}

  This thesis develops the \rrtfunnel{} algorithm, which is a provably robust
  feedback motion planner for nonlinear dynamical systems in the face of bounded
  uncertainties. This is done through the mathematical theory of convex
  optimization in order to search for Lyapunov functions which verify the
  regions of stability for the smooth nonlinear system at hand. These verified
  regions of stability is then employed as motion primitives for a global
  \ac{RRT} motion planning algorithm.

  More specifically, the \ac{SOS} programming theory is used to verify the
  regions of the configuration space surrounding a base set of trajectories that
  are guaranteed to give safe traversal of the planning environment.

  The verified trajectories are then employed as local motion primitives, each
  with its own, locally valid \ac{LQR} controller, which are then combined into
  a tree of controllers by the \ac{RRT} motion planner. With each controller
  having a verified region of the state space, these verified regions of
  traversal are referred to as 'funnels', and are combined together, giving a
  certificate of safe traversal. If the given funnels can be combined into a
  path from the initial state to the goal state, the plan is guaranteed to be
  collision free.

  The safe traversal guarantees of the algorithm is verified by running a first
  order unicycle model of an airplane through a simulated forest in order to
  verify that the robustness guarantees hold, even in the presence of noise and
  obstacles, which is shown not to hold for a \ac{RRT} algorithm which does not
  reason about disturbances. Here the \rrtfunnel{} algorithm is shown to
  significantly outperform an algorithm which does not handle uncertainty.


\end{abstract}
