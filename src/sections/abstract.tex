\chapter{Abstract}
% OPTIONAL SOLUTION:
% Use these settings if you are writing a monograph
% and have only one abstract.
% Do not use them if you have an abstract
% at the beginning of each chapter/paper.

\abstractintoc % Add abstract to Table of Contents
% \abstractnum   % Format abstract like a chapter

\begin{abstract}

  This thesis if focused on robust motion planning. Planning under uncertainty
  is necessary in order for an algorithm to move out of the lab and into the
  real world as in reality there are always an error in planning. Knowledge of
  position, the environment and the dynamics of the system are all uncertain to
  some degree. Sensory noise, tuning and readings may be off. Limited precision,
  and accidents may hinder the measurements and leave them with a constant error
  term. Thus in order for a planner to give guarantees on safe traversal through
  a real world environment, a motion planner needs to handle uncertainties.

  The \rrtfunnel{} algorithm is the answer provided to handle a subset of these
  problems. As it currently stands the algorithm handles uncertainty only in the
  position of the dynamical system, but can be easily extended to handle
  uncertainty in input and speed as well. Handling uncertainties in the
  environment will need a bit more work, but should be possible.

  The algorithm is built up through two main parts. One is the calculation of
  \textit{robust motion primitives}, which allows the global motion planner (in
  essence a regular \ac{RRT} algorithm) to remain completely oblivious to these
  difficulties, and hence act as though there were no uncertainties during the
  planning stage. This means that a lot of the difficulty of planning is handled
  during the off-line phase of generating the robust motion primitives
  themselves. This is done through the elegant theory of \ac{SOS} programming
  and verification. Through formulating the search for a \textit{Lyapunov
    function} for the system as a \ac{SOS} program, the trajectories are
  extended to so called 'funnels', which encorporate all the states the system
  can be in during a given time frame -- even in the face of uncertainties. In
  the literature this is referred to as a \textit{finite time reachable set},
  meaning that if the calculations are correct, it contains all the states that
  the system can evolve to given a set of initial conditions and a bounded
  uncertainty term.

  Later the aforementioned funnels are employed as \textit{motion primitives}
  for the motion planner. Which means that they all encode a discrete action.
  For the dynamical system in this thesis, which is a simple vehicle model, this
  means that a motion primitive can 'turn-left', 'turn-right', or 'go-straight'.
  Thus, by stacking one motion primtive after the other, one is able to create a
  plan, and hence build one long motion primitive through the overarching
  environment. With the motion primtives being robust to uncertainty as well,
  these plans are in theory guaranteed to be collision-free.

  This theoretical robustness guarantee is later put to the test in simulated
  experimental runs through a strip of forest of some density meant to make
  planning difficult, as the vehicle is equipped with a broken sensors, which
  always gives a drift in one dimension the world frame. Hence the vehicle is
  constantly faced with a positional reading which is not correct, and is forced
  to deal with this as best as it is able to during execution of the plan. As is
  shown, the robustness guarantees provided by the Lyapunov functions found
  through the \ac{SOS} programs formulated do provide safe traversal through the
  environment, as opposed to a planner which does not.

\end{abstract}